%% Sample LaTeX for CS427/519
%% Mike Rosulek
%% last update 2017-01-23

\usepackage{xspace,graphicx,amsmath,amssymb,xcolor}

%% operators

\newcommand{\pct}{\mathbin{\%}}
% makes ":=" aligned better
\usepackage{mathtools}
\mathtoolsset{centercolon}

% indistinguishability operator
% http://tex.stackexchange.com/questions/22168/triple-approx-and-triple-approx-with-a-straight-middle-line
\newcommand{\indist}{  \mathrel{\vcenter{\offinterlineskip
  \hbox{$\sim$}\vskip-.35ex\hbox{$\sim$}\vskip-.35ex\hbox{$\sim$}}}}
\renewcommand{\cong}{\indist}



\newcommand{\K}{\mathcal{K}}
\newcommand{\M}{\mathcal{M}}
\newcommand{\C}{\mathcal{C}}
\newcommand{\Z}{\mathbb{Z}}

\newcommand{\Enc}{\text{\sf Enc}}
\newcommand{\Dec}{\text{\sf Dec}}
\newcommand{\KeyGen}{\text{\sf KeyGen}}

% fancy script L
\usepackage[mathscr]{euscript}
\renewcommand{\L}{\ensuremath{\mathscr{L}}\xspace}
\newcommand{\lib}[1]{\ensuremath{\L_{\textsf{#1}}}\xspace}


\newcommand{\myterm}[1]{\ensuremath{\text{#1}}\xspace}
\newcommand{\bias}{\myterm{bias}}
\newcommand{\link}{\diamond}
\newcommand{\subname}[1]{\ensuremath{\textsc{#1}}\xspace}



%% colors
\definecolor{highlightcolor}{HTML}{F5F5A4}
\definecolor{highlighttextcolor}{HTML}{000000}
\definecolor{bitcolor}{HTML}{a91616}


%%% boxes for writing libraries/constructions
\usepackage{varwidth}

\newcommand{\codebox}[1]{%
        \begin{varwidth}{\linewidth}%
        \begin{tabbing}%
            ~~~\=\quad\=\quad\=\quad\=\kill % initialize tabstops
            #1
        \end{tabbing}%
        \end{varwidth}%
}
\newcommand{\titlecodebox}[2]{%
    \fboxsep=0pt%
    \fcolorbox{black}{black!10}{%
        \begin{varwidth}{\linewidth}%
        \centering%
        \fboxsep=3pt%
        \colorbox{black!10}{#1} \\
        \colorbox{white}{\codebox{#2}}%
        \end{varwidth}%
    }
}
\newcommand{\fcodebox}[1]{%
    \framebox{\codebox{#1}}%
}
\newcommand{\hlcodebox}[1]{%
    \fcolorbox{black}{highlightcolor}{\codebox{#1}}%
}
\newcommand{\hltitlecodebox}[2]{%
    \fboxsep=0pt%
    \fcolorbox{black}{black!15!highlightcolor}{%
        \begin{varwidth}{\linewidth}%
        \centering%
        \fboxsep=3pt%
        \colorbox{black!15!highlightcolor}{\color{highlighttextcolor}#1} \\
        \colorbox{highlightcolor}{\color{highlighttextcolor}\codebox{#2}}%
        \end{varwidth}%
    }
}


%% highlighting
\newcommand{\basehighlight}[1]{\colorbox{highlightcolor}{\color{highlighttextcolor}#1}}
\newcommand{\mathhighlight}[1]{\basehighlight{$#1$}}
\newcommand{\highlight}[1]{\raisebox{0pt}[-\fboxsep][-\fboxsep]{\basehighlight{#1}}}
\newcommand{\highlightline}[1]{%\raisebox{0pt}[-\fboxsep][-\fboxsep]{
    \hspace*{-\fboxsep}\basehighlight{#1}%
%}
}

%% bits
\newcommand{\bit}[1]{\textcolor{bitcolor}{\texttt{\upshape #1}}}
\newcommand{\bits}{\{\bit0,\bit1\}}


% \begin{document}

%     \section*{Example \LaTeX{} stuff}

%     \[
%         A \link
%         \fcodebox{
%             \underline{$\subname{challenge}(m_1 \cdots m_\ell)$:} \\
%             \> \highlightline{$r := \subname{samp}()$} \\
%             \> $c_0 := r$ \\
%             \> for $i = 1$ to $\ell$: \\
%             \>\> $r := \subname{query}(r)$ \\
%             \>\> $c_i := r \oplus m_i$ \\
%             \> return $c_0 c_1 \cdots c_\ell$
%         }
%         \link
%         \fcodebox{
%             $T := $ empty \\
%             \\
%             \underline{$\subname{query}(r)$:} \\
%             \> if $T[r]$ undefined: \\
%             \>\> \highlightline{$T[r] := \subname{samp}()$} \\
%             \> return $T[r]$
%         }
%         \link
%         \hlcodebox{
%             \underline{$\subname{samp}()$:} \\
%             \> $s \gets \bits^\lambda$ \\
%             \> return $s$
%         }
%     \]

%     \[
%             \titlecodebox{$\lib{cpa-L}^\Sigma$}{
%                 $k \gets \Sigma.\KeyGen$ \\[8pt]

%                 \underline{$\subname{challenge}(m_L, m_R)$:} \\
%                 \> $c := \Sigma.\Enc(k,\mathhighlight{m_L})$ \\
%                 \> return $c$
%             }
%     \]
% %
% \begin{align*}
%     \lib{left} \equiv \lib{right}
%         &\Leftrightarrow \forall A: \Pr[A \link \lib{left} \mbox{ outputs 1}] = \Pr[A \link \lib{right} \mbox{ outputs 1}]
% \\
%     \lib{left} \indist \lib{right}
%         &\Leftrightarrow \forall \mbox{ poly-time } A: \Pr[A \link \lib{left} \mbox{ outputs 1}] \approx \Pr[A \link \lib{right} \mbox{ outputs 1}]
% \end{align*}

%      \begin{center}
%      \framebox{
%         \codebox{
%             \underline{$\KeyGen$:} \\
%             \> $k \gets \bits^\lambda$ \\
%             \> return $k$
%         }
%         \qquad
%         \codebox{
%             \underline{$\Enc(k,m)$:} \\
%             \> $r \gets \bits^\lambda$\\
%             \> $x := F(k,r) \oplus m$ \\
%             \> return $(r,x)$
%         }
%         \qquad
%         \codebox{
%             \underline{$\Dec(k,(r,x))$:} \\
%             \> $m := F(k,r) \oplus x$ \\
%             \> return $m$
%         }
%     }

%     \bigskip

%     \fcodebox{
%         \underline{$H(s)$:} \\
%         \> $x := G(s)$ \\
%         \> $y := G(x_{\textsf{right}})$ \\
%         \> return $x_{\textsf{left}} \| y$
%     }
%     \end{center}



% \end{document}
